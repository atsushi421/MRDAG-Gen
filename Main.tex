\documentclass[conference]{IEEEtran}
\IEEEoverridecommandlockouts

\usepackage{cite}
\usepackage{amsmath,amssymb,amsfonts}
\usepackage{amsthm}
\usepackage{algorithmic}
\usepackage{textcomp}
\usepackage{xcolor}
\usepackage{graphics}
\usepackage{textcomp}
\usepackage{comment}
\usepackage{tabularx}
\usepackage{breqn}
\usepackage{url}
\usepackage{listings}
\usepackage{threeparttable}
\usepackage[linesnumbered,ruled]{algorithm2e}
\usepackage{color}
\usepackage{soul}
\usepackage{multirow}

\setcounter{topnumber}{100}
\setcounter{bottomnumber}{100}
\setcounter{totalnumber}{100}
\renewcommand{\textfraction}{0.0}
\renewcommand{\topfraction}{1.0}

% code style setting
\lstset{
    %プログラム言語(複数の言語に対応,C,C++も可)
    language = c,
    %枠外に行った時の自動改行
    breaklines = true,
    %自動開業後のインデント量(デフォルトでは20[pt])	
    breakindent = 5pt,
    %標準の書体
    basicstyle = \ttfamily\scriptsize,
    %basicstyle = {\small}
    %コメントの書体
    commentstyle = {\itshape \color[cmyk]{1,0.4,1,0}},
    %関数名等の色の設定
    % classoffset = 0,
    %キーワード(int, ifなど)の書体
    keywordstyle = {\bfseries \color[cmyk]{0,1,0,0}},
    %""で囲まれたなどの"文字"の書体
    % stringstyle = {\ttfamily \color[rgb]{0,0,1}},
    %枠 "t"は上に線を記載, "T"は上に二重線を記載
    %他オプション:leftline,topline,bottomline,lines,single,shadowbox
    frame = TB,
    %frameまでの間隔(行番号とプログラムの間)
    framesep = 3pt,
    %行番号の位置
    numbers = left,
    %行番号の間隔
    stepnumber = 1,
    %右マージン
    xrightmargin=0pt,
    %左マージン
    xleftmargin=0pt,
    %行番号の書体
    numberstyle = \scriptsize,
    %タブの大きさ
    tabsize = 4,
    %キャプションの場所("tb"ならば上下両方に記載)
    captionpos = t,
    columns=[l]fullflexible,
    linewidth=6cm,
    escapeinside={(*}{*)}
}

\bibliographystyle{unsrt}

\def\BibTeX{{\rm B\kern-.05em{\sc i\kern-.025em b}\kern-.08em
    T\kern-.1667em\lower.7ex\hbox{E}\kern-.125emX}}


%%% Original command %%%
\sethlcolor{yellow}
\newcommand{\todo}[1]{\hl{\textbf{TODO:} #1}}
\newcommand{\hack}[1]{\hl{\textbf{HACK:} #1}}
% \newcommand{\todo}[1]{}
% \newcommand{\hack}[1]{}


%%% Original function %%%
\def\MR#1#2{\multirow{#1}{*}{#2}}
\def\MC#1#2#3{\multicolumn{#1}{#2}{#3}}


%%% Start paper %%%
\begin{document}

\title{\todo{}}
\maketitle

\author{
    \IEEEauthorblockN{Atsushi Yano and Takuya Azumi}
    \IEEEauthorblockA{\textit{Graduate School of Science and Engineering,} \\ \textit{Saitama University}}
}

\begin{abstract}
    Real-time systems include various requirements such as deadline and resource constraints.
    Such systems are becoming larger and more complex, and studies on performance analyses and efficient scheduling algorithms are becoming increasingly important.
    Directed acyclic graph (DAG) models, which can express task dependencies and parallelism, are used for such studies.
    Random DAG sets are used to demonstrate the effectiveness and objectivity of methods proposed for real-time systems.
    However, there is no random DAG generation tool available that can generate a DAG set that considers the latest multi-rate applications.
    Therefore, researchers need to generate random DAG sets on their own, leading to additional effort and reduced reliability and reproducibility.
    To solve this problem, we propose a random DAG generator considering multi-rate applications for reproducible scheduling evaluation (RD-Gen).
    RD-Gen also enables batch generation of random DAG sets with different parameters.
    Case studies are used to demonstrate that RD-Gen can manage various problem settings and DAG study requirements.
\end{abstract}


\begin{IEEEkeywords}
    \todo{}
\end{IEEEkeywords}

\section{Introduction}
\label{sec: introduction}

\cite{10.1145/3381847}
\todo{}


\section{System model}
\label{sec: system_model}


\begin{figure}[tb]
    \centering
    \scalebox{1.0}{
        \includegraphics[width=\linewidth]{./src/figure/system_model.pdf}
    }
    \caption{System model.}
    \label{fig: system_model}
\end{figure}


\begin{table}[tb]
    \centering
    \caption{DAG Notations}
    \label{tab: dag_notations}
    \renewcommand{\arraystretch}{1.2}
    \scalebox{1.0}{
        \begin{tabular}{c|l}\hline\hline
            Symbols        & \MC{1}{c}{Descriptions}                            \\\hline\hline
            $G$            & DAG                                                \\ \hline
            $V$            & Set of all nodes of $G$                            \\ \hline
            $|V|$          & Total number of nodes of $G$                       \\ \hline
            $E$            & Set of all edges of $G$                            \\ \hline
            $\tau_i$       & $i$-th node                                        \\ \hline
            $C_i$          & Worst-case execution time (WCET) of $\tau_i$       \\ \hline
            $e_{i,j}$      & Edge between $\tau_i$ and $\tau_j$                 \\ \hline
            $comm_{i,j}$   & Communication time of $e_{i,j}$                    \\\hline
            $CCR$          & communication-to-computation ratio (CCR) of $G$    \\\hline
            $D$            & End-to-end deadline                                \\ \hline
            $\tau^{tm}_i$  & $i$-th timer-driven node                           \\ \hline
            $\phi_i$       & Offset of $\tau^{tm}_i$                            \\ \hline
            $T_i$          & Execution period of $\tau^{tm}_i$                  \\ \hline
            $d_i$          & Relative deadline of $\tau^{tm}_i$                 \\ \hline
            $u_i$          & Utilization of $\tau^{tm}_i$                       \\\hline
            $U$            & Total utilization of $G$                           \\\hline
            $\Gamma_i$     & $i$-th chain                                       \\\hline
            $|\Gamma|$     & Total number of chains of $G$                      \\\hline
            $C_{\Gamma_i}$ & WCET of $\Gamma_i$                                 \\ \hline
            $u_{\Gamma_i}$ & Utilization of $\Gamma_i$                          \\ \hline
            $T_{\Gamma_i}$ & Period of the head timer-driven node of $\Gamma_i$ \\ \hline
        \end{tabular}
    }
\end{table}


This section presents the system model, as shown in Fig.~\ref{fig: system_model}.
Section~\ref{ssec: single_rate_dag} describes a basic single-rate DAG.
Section~\ref{ssec: multi_rate_dag} explains a multi-rate DAG.
The DAG notation used in this paper is given in Table~\ref{tab: dag_notations}.


\subsection{Single-rate DAG}
\label{ssec: single_rate_dag}

Single-rate DAGs are DAGs with either a single entry node or, all entry nodes entering at the same time; such DAGs can be used in real-time applications \cite{zhao2020dag}, cyber-physical systems \cite{senapati2021hmds}, and cloud computing \cite{kaur2020deep}.
Here, the entry node represents the input to the system (e.g., a sensor event or a command from the user), and the exit node represents the final output from the system.

A DAG consists of a node set and an edge set, denoted $G = (V, E)$.
Nodes represent tasks in the system, and edges represent communication and dependencies between nodes or priority constraints.
$V$ is the set of all nodes, expressed as $V = \{\tau_1, ..., \tau_{|V|}\}$, where $|V|$ is the total number of nodes.
Each node has a worst-case execution time (WCET), and the WCET of $\tau_i$ is denoted as $C_i$.
$E$ is the set of all edges, where each edge $e_{i,j} \in E$ represents communication between $\tau_i$ and $\tau_j$ and a priority constraint.
When $e_{i,j}$ exists in the DAG, $\tau_j$ cannot be executed until $\tau_i$ has completed its execution and the output of $\tau_i$ has arrived.
If the communication time is given as an assumption, the communication time at $e_{i,j}$ is denoted as $comm_{i, j}$.
The ratio of the sum of the communication times of all edges to the sum of the execution times of all nodes is called the communication-to-computation ratio (CCR) and is defined by Eq.~(\ref{eq: ccr}).

\begin{equation}
    \label{eq: ccr}
    CCR = \frac{\sum\limits_{e_{i,j} \in E}comm_{i, j}}{\sum\limits_{\tau_i \in V}C_i}
\end{equation}

An end-to-end deadline $D$ is set at the exit node when it is necessary to guarantee the safety of hard real-time systems \cite{yano2021work} or the cloud computing quality of service \cite{zhang2020efficient}.


\subsection{Multi-rate DAG}
\label{ssec: multi_rate_dag}

A multi-rate DAG is a DAG that contains multiple nodes that are triggered with different periods.
Here, the definitions of nodes and edges in a multi-rate DAG are the same as those given in Section~\ref{ssec: single_rate_dag}.
Multi-rate DAGs can be broadly classified into two categories: (i) DAGs in which all nodes are timer driven, such as in automotive systems \cite{kordon2020evaluation, verucchi2020latency}, and (ii) DAGs that combine a chain consisting of timer-driven nodes and a sequence of linked event-driven nodes, such as in self-driving systems \cite{choi2021picas, tang2020response}.


\subsubsection{Multi-rate DAG consisting of only timer-driven nodes}
\label{sssec: dag_only_timer}

Each timer-driven node in such multi-rate DAGs is denoted by $\tau^{tm}_i$, and $\tau^{tm}_i$ is characterized by the tuple $(\phi_i, Ci, Ti, di)$, where $\phi_i$, $T_i$, and $d_i$ represent the offset, execution period, and relative deadline, respectively.
For DAGs that consider timer-driven nodes, every timer-driven node has a relative deadline of $T_i$ time units indicating that every job of $\tau^{tm}_i$ has an absolute deadline at $T_i$ time units after its release \cite{yang2020mixed, cho2021conditionally}.
Such a time constraint is called an implicit deadline.

The utilization of $\tau^{tm}_i$ is denoted as $u_i$, such that $u_i = C_i / T_i$.
The total utilization $U$ of a DAG consisting only of timer-driven nodes is defined by Eq.~(\ref{eq: total_utilization_only_timer}).

\begin{equation}
    \label{eq: total_utilization_only_timer}
    U = \sum_{\tau^{tm}_i \in V}u_i
\end{equation}


\begin{figure}[tb]
    \centering
    \scalebox{1.0}{
        \includegraphics[width=\linewidth]{./src/figure/chain_based_dag.pdf}
    }
    \caption{Multi-rate DAGs consisting of multiple chains.}
    \label{fig: chain_dag}
\end{figure}


\subsubsection{Chain-based multi-rate DAG}
\label{sssec: dag_chain}

In the latest multi-rate applications, DAGs consisting of the multiple chains shown in Fig.~\ref{fig: chain_dag} are considered.
Each chain $\Gamma_i$ is denoted as $\Gamma_i = \{\tau^{tm}_i, \tau_k, ..., \tau_{|\Gamma_i|}\}$, where $|\Gamma_i|$ is the number of nodes that compose $\Gamma_i$.
The head $\tau^{tm}_i$ in the chain is triggered periodically, and subsequent event-driven nodes are triggered by their direct predecessors.
This definition is the same as that used in existing studies \cite{choi2020chain, tang2020response}.
The WCET of $\Gamma_i$ is denoted by $C_{\Gamma_i}$ and defined in Eq.~(\ref{eq: wcet_chain}).

\begin{equation}
    \label{eq: wcet_chain}
    C_{\Gamma_i} = \sum_{\tau_i \in \Gamma}C_i
\end{equation}

Because the chain is executed in a manner dependent on the period of the head timer-driven node, the utilization of the chain $u_{\Gamma_i}$ is defined by Eq.~(\ref{eq: chain_utilization}).

\begin{equation}
    \label{eq: chain_utilization}
    u_{\Gamma_i} = \frac{C_{\Gamma_i}}{T_{\Gamma_i}}
\end{equation}

\noindent Here, $T_{\Gamma_i}$ is the period of the head timer-driven node $\tau^{tm}_i$ of the chain.
The total utilization of the chain-based multi-rate DAG is defined by Eq.~(\ref{eq: chain_total_utilization}).

\begin{equation}
    \label{eq: chain_total_utilization}
    U = \sum_{\Gamma_i \in V}u_{\Gamma_i}
\end{equation}

Chain-based multi-rate DAGs primarily exist in robot operating system (ROS)-based systems \cite{casini2019response, choi2021picas}.
In a typical ROS-based system, such as a self-driving system (e.g., Autoware \cite{future}), different sensor data are processed and merged by multiple chains to output the final command.
When modeling ROS-based systems as DAGs, it is necessary to consider DAGs in which multiple chains merge at the exit nodes ((a) in Fig.~\ref{fig: chain_dag}), where the chain branches ((b) in Fig.~\ref{fig: chain_dag}), and those in which multiple chains are vertically linked ((c) in Fig.~\ref{fig: chain_dag}).
RD-Gen can generate all of these DAG types by using various parameters.


\section{Design and Implementation}
\label{sec: design_implementation}

\todo{}


\section{Case study}
\label{sec: case_study}


This section illustrates that MRDAG-Gen can generate the random DAG sets used in the evaluation of existing studies based on DAGs.
Case studies of a single-rate DAG, a multi-rate DAG consisting of only timer-driven nodes, and a chain-based multi-rate DAG are shown, respectively.


\subsection{Case study 1}
\label{ssec: case_study_1}


\begin{figure*}[t]
    \begin{tabular}{c}
        %1
        \begin{minipage}{0.35\linewidth}
            \begin{flushleft}
                \begin{tabular}{l}
                    \lstset{linewidth=5.5cm, basicstyle=\scriptsize}
                    \lstinputlisting{src/code/case_study_single.txt}
                \end{tabular}
            \end{flushleft}
            \centering
            Input YAML file
        \end{minipage}
        % 2
        \begin{minipage}{0.22\linewidth}
            \centering
            \scalebox{1.0}{\includegraphics[keepaspectratio, width = \linewidth]{./src/figure/case_study_single/ccr_0.1.pdf}}
            \hspace{3.0cm} Number of nodes: 10, CCR: 0.1
        \end{minipage}
        % 3
        \begin{minipage}{0.47\linewidth}
            \centering
            \scalebox{0.8}{\includegraphics[keepaspectratio, width = \linewidth]{./src/figure/case_study_single/ccr_1.0.pdf}}
            \hspace{3.5cm} Number of nodes: 10, CCR: 1.0 \\
            \vspace{3mm}
            \centering
            \scalebox{0.7}{\includegraphics[keepaspectratio, width = \linewidth]{./src/figure/case_study_single/ccr_10.0.pdf}}
            \hspace{3.0cm} Number of nodes: 10, CCR: 10.0
        \end{minipage}
    \end{tabular}
    \centering
    \caption{Example of the generation a random DAG set with varying CCR, number of nodes, execution time, and communication time using the {\it Fan-in/Fan-out} method.}
    \label{fig: case_study_single}
\end{figure*}


Since CCR changes the nature of DAGs and affects the performance of scheduling algorithms, existing studies of single-rate DAGs have used random DAG sets with different CCR values in their evaluations.
An example of the generation in MRDAG-Gen of a random DAG set with varying CCR, number of nodes, execution time, and communication time using the {\it Fan-in/Fan-out} method as used in existing studies \cite{subbaraj2020multi, liu2016minimizing, sheikh2016sixteen} is shown in Fig.~\ref{fig: case_study_single}.
Since the {\it Combination} is specified for the {\it Number of nodes} (lines 6-7 in Fig.~\ref{fig: case_study_single}) and {\it CCR} (lines 21-22 in Fig.~\ref{fig: case_study_single}), MRDAG-Gen generates 100 random DAGs each for all combinations of these parameters (i.e., $\{10, 20, 30, ..., 1000\} \times \{0.1, 0.2, 0.5, 1.0, 2.0, 5.0, 10.0\}$).
For each DAG, after graph construction, execution time is randomly assigned to each node in the range of 1 to 30 (lines 19-20 in Fig.~\ref{fig: case_study_single}).
Then, the total communication time is calculated from the CCR and the sum of execution time based on Eq.~\ref{eq: ccr}, and the total communication time is randomly distributed to each edge.
Thus, the user can generate all DAG sets used in the evaluation with a single command, without adjusting the number of nodes or CCR.


\subsection{Case study 2}
\label{ssec: case_study_2}


\begin{figure*}[t]
    \begin{tabular}{c}
        %1
        \begin{minipage}{0.30\linewidth}
            \begin{flushleft}
                \begin{tabular}{l}
                    \lstset{linewidth=4.0cm, basicstyle=\scriptsize}
                    \lstinputlisting{src/code/case_study_all_timer.txt}
                \end{tabular}
            \end{flushleft}
            \centering
            Input YAML file
        \end{minipage}
        % 2
        \begin{minipage}{0.28\linewidth}
            \centering
            \scalebox{1.0}{\includegraphics[keepaspectratio, width = \linewidth]{./src/figure/case_study_all_timer/u_0.05.pdf}}
            \hspace{3.0cm} Number of nodes: 10,\\ Total utilization: 0.05
        \end{minipage}
        % 3
        \begin{minipage}{0.45\linewidth}
            \centering
            \scalebox{0.58}{\includegraphics[keepaspectratio, width = \linewidth]{./src/figure/case_study_all_timer/u_0.5.pdf}}
            \hspace{3.5cm} Number of nodes: 10, Total utilization: 0.5 \\
            \vspace{3mm}
            \centering
            \scalebox{0.75}{\includegraphics[keepaspectratio, width = \linewidth]{./src/figure/case_study_all_timer/u_0.95.pdf}}
            \hspace{3.0cm} Number of nodes: 10, Total utilization: 0.95
        \end{minipage}
    \end{tabular}
    \centering
    \caption{Example of generating a multi-rate DAG consisting of only timer-driven nodes of various total utilization.}
    \label{fig: case_study_all_timer}
\end{figure*}


Random DAG sets with different total utilization are used in the evaluation of most studies considering multi-rate DAGs.
An example of generating a multi-rate DAG with a random period and execution time for each node based on the total utilization of the DAG, as used in existing studies \cite{he2021response, gunzel2021suspension, ueter2021hard}, is shown in Fig~\ref{fig: case_study_all_timer}.
Here, timer-driven nodes are drawn as squares.
Since {\it Combination} is specified for {\it Total utilization} (lines 21-22 in Fig~\ref{fig: case_study_all_timer}), MRDAG-Gen generates 100 DAGs each with total utilization from 5\% to 95\% in 5\% increments.
For each DAG, the utilization of each node is determined by the UUniFast method to satisfy the total utilization, and the execution time is calculated from the utilization and a randomly selected period from a uniform distribution ranging from 1 to 100.


\subsection{Case study 3}
\label{ssec: case_study_3}


\begin{figure*}[t]
    \begin{tabular}{c}
        %1
        \begin{minipage}{0.25\linewidth}
            \begin{flushleft}
                \begin{tabular}{l}
                    \lstset{linewidth=4.5cm, basicstyle=\scriptsize}
                    \lstinputlisting{src/code/case_study_chain.txt}
                \end{tabular}
            \end{flushleft}
            \centering
            Input YAML file
        \end{minipage}
        % 2
        \begin{minipage}{0.32\linewidth}
            \centering
            \scalebox{0.7}{\includegraphics[keepaspectratio, width = \linewidth]{./src/figure/case_study_chain/u_0.5.pdf}}
            \hspace{3.0cm} Number of chains: 4,\\ Total utilization: 0.5, \\ Number of exit nodes: 3
        \end{minipage}
        % 3
        \begin{minipage}{0.36\linewidth}
            \centering
            \scalebox{1.0}{\includegraphics[keepaspectratio, width = \linewidth]{./src/figure/case_study_chain/u_4.0.pdf}}
            \hspace{3.5cm} Number of chains: 7, Total utilization: 4.0 \\ Number of exit nodes: 3
        \end{minipage}
    \end{tabular}
    \centering
    \caption{Example of generating a multi-rate DAG consisting of only timer-driven nodes of various total utilization.}
    \label{fig: case_study_chain}
\end{figure*}


In the evaluation of chain-based multi-rate DAGs, a random DAG set is used with varying total utilization for the entire system and each chain.
An example of generating a random DAG set, as used in existing studies \cite{tang2020response, choi2021picas}, in which the utilization of each chain is determined from the total utilization of the entire system, and the period and execution time of each node is randomly assigned to satisfy the utilization of the chain is shown in Fig.~\ref{fig: case_study_chain}.
MRDAG-Gen randomly sets the utilization of each chain using the UUniFast method to achieve a determined total utilization rate.
Here, since {\it Maximum utilization} is specified as 1.0 (line 24 in Fig.~\ref{fig: case_study_chain}), MRDAG-Gen never generates a chain with a utilization greater than 1.0.
The period of each chain is randomly set in the range of 50 to 1000, and the execution time of each node is calculated based on the utilization and period (lines 10-18 in Algorithm~\ref{alg: set_utilization}).


\section{Evaluation}
\label{sec: evaluation}


\begin{table}[tb]
    \centering
    \caption{Processes that must be implemented by the user \\ and lines of code}
    \label{tab: processes}
    \renewcommand{\arraystretch}{1.2}
    \scalebox{0.8}{
        \begin{tabular}{c|l|c}\hline\hline
            Symbols & \MC{1}{c|}{Processes}                                         & Lines of code \\\hline\hline
            $(a)$   & Loop with different parameter values                          & 3             \\\hline
            $(b)$   & Separate DAGs generated for each parameter into directories   & 5             \\\hline
            $(c)$   & Load .tgff file as DAG                                        & 40            \\\hline
            $(d)$   & Construct a DAG using G(n, p) method                          & 10            \\\hline
            $(e)$   & Construct a DAG using chain-based method                      & 80            \\\hline
            $(f)$   & Merge the exit nodes to a specific number                     & 10            \\\hline
            $(g)$   & Add edges to be weakly connected                              & 15            \\\hline
            $(h)$   & Set the properties randomly within a specific range           & 10            \\\hline
            $(i)$   & \tabml{Set the execution time for each node and                               \\ the communication time for each edge based on the CCR}                             & 25            \\\hline
            $(j)$   & \tabml{Set the utilization for a multi-rate DAG consisting of                 \\ only timer-driven nodes (lines 1-7 in Algorithm~\ref{alg: set_utilization})}                                             & 30            \\\hline
            $(k)$   & \tabml{Set the utilization for a chain-based multi-rate DAG                   \\ (lines 8-19 in Algorithm~\ref{alg: set_utilization})}                                             & 50            \\\hline
        \end{tabular}
    }
\end{table}


This section shows that compared to existing random DAG generation tools, RD-Gen can generate a random DAG set with fewer lines of code and no unique user implementation.
Table~\ref{tab: processes} shows the processes that must be implemented by the user and the number of lines of code.
Here, the process in Table~\ref{tab: processes} is written in Python and shell scripts, and the Python NetworkX library is used for processes related to DAGs.


\begin{table}[tb]
    \centering
    \caption{Effort required to generate random DAG sets \\ for {\it Case Study 1} \cite{subbaraj2020multi, liu2016minimizing, sheikh2016sixteen}}
    \label{tab: eval_case_study_single}
    \renewcommand{\arraystretch}{1.2}
    \scalebox{0.8}{
        \begin{tabular}{|c|c|l|}
            \hline
            \multicolumn{1}{|l|}{}         & \multicolumn{1}{c|}{\begin{tabular}[c]{@{}c@{}}Lines of description\\ for tool options\end{tabular}} & \multicolumn{1}{c|}{Lines of code}                  \\ \hline
            TGFF \cite{tgff}               & 6                                                                                                    & \tabml{$(a) \times 2 + (b) + (c) + (f) + (g) + (i)$ \\ $= 101$} \\ \hline
            GGen \cite{cordeiro2010random} & 3                                                                                                    & \tabml{$(a) \times 2 + (b) + (g) + (i)$             \\ $= 51$}              \\ \hline
            RD-Gen                         & 20                                                                                                   & 0                                                   \\ \hline
        \end{tabular}
    }
\end{table}


The amount of description required to generate a random DAG set for the case studies in Section~\ref{ssec: case_study_1} is shown in Table~\ref{tab: eval_case_study_single}.
In this case, while TGFF and GGen require the user to implement more than 50 lines, RD-Gen can generate all random DAG sets using only the tool's functionality.
Since TGFF and GGen do not have the functionality to generate DAGs with different parameter settings at once, the user must execute the generation command many times while changing the parameters, using shell scripts or other means.
In contrast, RD-Gen generates DAGs with all parameter combinations by specifying {\it Combination} as a parameter and automatically divides them into directories.
Although TGFF uses the {\it Fan-in/Fan-out} method to construct DAGs, the .tgff files output by TGFF are in a proprietary format and require users to implement a reading process.
Furthermore, since TGFF and GGen do not guarantee that the graph is weakly connected, users may unintentionally use a non weakly connected DAG for evaluation.
TGFF and GGen cannot specify parameters such as CCR that are calculated from multiple properties. Therefore, the CCR must be adjusted for the DAG set after it is generated.
In RD-Gen, users can automate all of the above processes by inputting the YAML file as shown on the left side of Fig.~\ref{fig: case_study_single}.


\begin{table}[tb]
    \centering
    \caption{Effort required to generate random DAG sets \\ for {\it Case Study 2} \cite{he2021response, gunzel2021suspension, ueter2021hard}}
    \label{tab: eval_case_study_all_timer}
    \renewcommand{\arraystretch}{1.2}
    \scalebox{0.8}{
        \begin{tabular}{|c|c|l|}
            \hline
            \multicolumn{1}{|l|}{}         & \multicolumn{1}{c|}{\begin{tabular}[c]{@{}c@{}}Lines of description\\ for tool options\end{tabular}} & \multicolumn{1}{c|}{Lines of code}   \\ \hline
            TGFF \cite{tgff}               & -                                                                                                    & \tabml{$(a) + (b) + (d) + (g) + (j)$ \\ $= 63$} \\ \hline
            GGen \cite{cordeiro2010random} & 2                                                                                                    & \tabml{$(a) + (b) + (g) + (j)$       \\ $= 53$}       \\ \hline
            RD-Gen                         & 20                                                                                                   & 0                                    \\ \hline
        \end{tabular}
    }
\end{table}


The effort required to generate a random DAG set for the case studies in Section~\ref{ssec: case_study_2} is shown in Table~\ref{tab: eval_case_study_all_timer}.
TGFF does not support the {\it G(n, p)} method and multi-rate DAG.
GGen allows properties to be randomly set to nodes or edges within a user-specified range after graph construction.
However, GGen does not allow users to specify the total utilization used in the evaluation of most studies that consider multi-rate DAGs.
Therefore, the user has to go through the trouble of implementing such as lines 1-7 in Algorithm~\ref{alg: set_utilization}.
In contrast, RD-Gen uses the UUniFast method to randomly and automatically set the period and execution time for each node to meet the specified total utilization.


\begin{table}[tb]
    \centering
    \caption{Effort required to generate random DAG sets \\ for {\it Case Study 3} \cite{tang2020response, choi2021picas}}
    \label{tab: eval_case_study_chain}
    \renewcommand{\arraystretch}{1.2}
    \scalebox{0.8}{
        \begin{tabular}{|c|c|l|}
            \hline
            \multicolumn{1}{|l|}{}         & \multicolumn{1}{c|}{\begin{tabular}[c]{@{}c@{}}Lines of description\\ for tool options\end{tabular}} & \multicolumn{1}{c|}{Lines of code} \\ \hline
            TGFF \cite{tgff}               & -                                                                                                    & \tabml{$(a) + (b) + (d) + (k)$     \\ $= 138$}      \\ \hline
            GGen \cite{cordeiro2010random} & -                                                                                                    & \tabml{$(a) + (b) + (d) + (k)$     \\ $= 138$}      \\ \hline
            RD-Gen                         & 22                                                                                                   & 0                                  \\ \hline
        \end{tabular}
    }
\end{table}


The amount of description to generate a chain-based random DAG set, as in the case study in Section~\ref{ssec: case_study_3}, is shown in Table~\ref{tab: eval_case_study_chain}.
Chain-based multi-rate DAGs cannot be generated by TGFF and GGen because they are considered in state-of-the-art self-driving systems research.
RD-Gen can generate a batch of random chain-based DAG sets with different total utilization without any user implementation by inputting a YAML file as shown in Fig.~\ref{fig: case_study_chain}.

The evaluation results demonstrated that RD-Gen can generate the random DAG sets used in the evaluation with only an intuitive YAML file description, regardless of the single rate multi rate.
Therefore, RD-Gen is a flexible evaluation platform that can be adapted to various requirements and provides reliability and reproducibility for the latest DAG studies.


\section{Related work}
\label{sec: related_work}

This section describes existing random DAG generation tools and existing studies using random DAGs and compares them to MRDAG-Gen.
Table~\ref{tab: comparison} compares MRDAG-Gen and existing methods.

\begin{table}[tb]
    \centering
    {

        \caption{MRDAG-Gen vs existing methods}
        \label{tab: comparison}
        \renewcommand{\arraystretch}{1.2}
        \scalebox{0.8}{
            \begin{tabular}{c|cccccc} \hline\hline
                                                           & RMD & RPU & RDT & RTE & OCG \\\hline
                DAGEN \cite{amalarethinam2011dagen}        &     &     & \ch &     &     \\\hline
                MRTG \cite{ashish2016modular}              &     &     & \ch &     &     \\\hline
                TGFF \cite{tgff}                           &     &     & \ch &     &     \\\hline
                GGen \cite{cordeiro2010random}             & \ch &     & \ch &     &     \\\hline
                Kordon et al. \cite{kordon2020evaluation}  & \ch &     &     &     &     \\\hline
                Yang et al.   \cite{yang2020mixed}         & \ch & \ch &     &     &     \\\hline
                Gunzel et al. \cite{gunzel2021suspension}  & \ch & \ch &     &     &     \\\hline
                Ueter et al. \cite{ueter2021hard}          & \ch & \ch &     &     &     \\\hline
                Dong et al. \cite{dong2019efficient}       & \ch & \ch &     &     &     \\\hline
                He et al. \cite{he2021response}            & \ch & \ch &     &     &     \\\hline
                Voronov et al. \cite{voronov2021ai}        & \ch & \ch &     &     &     \\\hline
                Verucchi et al. \cite{verucchi2020latency} & \ch & \ch &     &     &     \\\hline
                Klaus et al. \cite{klaus2021constrained}   & \ch & \ch &     &     &     \\\hline
                Tang et al. \cite{tang2020response}        & \ch & \ch &     & \ch &     \\\hline
                Choi et al. \cite{choi2021picas}           & \ch & \ch &     & \ch &     \\\hline
                MRDAG-Gen                                  & \ch & \ch & \ch & \ch & \ch \\\hline
            \end{tabular}
        }
        \begin{tablenotes}[normal]{
                \item {RMD}: Random generation of multi-rate DAGs
                \item {RPU}: Random property settings based on total utilization
                \item {RDT}: Random DAG generation tool
                \item {RTE}: Random generation of chain-based multi-rate DAG
                \item {OCG}: One-command batch generation of random DAGs
            }
        \end{tablenotes}
    }
\end{table}


\subsection{Random DAG generation with unique implementation}
\label{sec: random_tool}

The random DAG generation tool provides reliability and reproducibility for the evaluation of scheduling and latency analysis studies.
TGFF \cite{tgff} is the first tool proposed for this purpose and has been used to evaluate the most recent studies \cite{roeder2021energy, fard2021analytical, wu2021evolutionary, costa2021extracting}.
TGFF can determine the shape of the DAG mainly by specifying the maximum and/or minimum input degree and maximum (first) output degree for one node ({\it Fan-in/Fan-out} method).
TGFF can quickly generate many DAGs, and the task set can be easily reproduced by other researchers by inputting the same parameters.
However, TGFF is a tool released in 1998 and has many problems, such as its output format (.tgff), which is difficult to handle and cannot generate the multi-rate DAGs.

GGen \cite{cordeiro2010random} is a unified implementation of the classical task graph generation methods used in the scheduling domain.
GGen allows the user to add properties such as execution period and communication time to nodes and edges after generating DAGs using a user-specified generation method.
However, GGen does not allow the specification of constraints between different properties, such as implicit deadlines (i.e., the execution time of a node must not exceed its period).
Therefore, users with such requirements must adjust the values themselves.

Other random DAG generation tools such as DAGGEN \cite{amalarethinam2011dagen} and MRTG \cite{ashish2016modular} have been proposed.
DAGGEN generates random workflow applications by specifying node load balancing, edge connection probability, and workflow shape.
MRTG is a random DAG generation tool with a module-based implementation for user extensibility.
However, these tools are not capable of generating multi-rate DAGs.
In contrast, MRDAG-Gen can flexibly generate multi-rate DAGs of various types.
In addition, MRDAG-Gen can automatically set properties calculated by multiple values such as CCR and total utilization.


\subsection{Unique implementation of random DAG generation}

This section presents existing studies in which the authors generate random DAG sets for evaluation in their own implemented algorithms and settings.
Since there are no random DAG generation tools capable of generating multi-rate DAGs as described in Section~\ref{sec: random_tool}, most studies that consider multi-rate DAGs have unique implementations.

In real-time systems such as in-vehicle systems and self-driving systems, multi-rate DAGs consisting of only timer-driven nodes are considered.
Many real-time researchers use the {\it G(n, p)} method to construct DAGs and generate random task sets with the different number of nodes, different total utilization, and different periods. \cite{voronov2021ai, he2021response, dong2019efficient}.
While proprietary algorithms are used to construct the graph in some cases, the approach is similar in that total utilization is determined using the UUniFast method and WCET of nodes are assigned based on utilization and periods \cite{yang2020mixed, gunzel2021suspension, ueter2021hard}.

In a multi-rate DAG considering automotive systems, a period is randomly assigned to each node based on the period observed in the automotive application.
Verucchi et al. \cite{verucchi2020latency} randomly extended the automotive benchmark proposed by BOSCH in the 2015 WATERS Challenge \cite{kramer2015real} to analyze the performance of the proposed method.
Verucchi et al. randomly set the task period from $[1, 5, 10, 20, 50, 100, 200, 1000]$ ms, as found in automotive applications, DAG utilization.
Klaus et al. \cite{klaus2021constrained} set the utilization, period, and the number of nodes for each node of the DAG.
Verucchi et al. and Klaus et al. create random task sets based on node chains consisting of only timer-driven nodes.
MRDAG-Gen can also cover such chains consisting of only timer-driven nodes by using {\it Chain-based} methods and specifying the {\it Period type} as {\it "All"}.
Kordon et al. \cite{kordon2020evaluation} randomly set the period, the number of edges per task, release time, and the number of entry nodes for DAGs generated by the Python networkX library.
MRDAG-Gen can automatically set all combinations of total utilization, periods, and the number of nodes as described above.

Studies of chain-based multi-rate DAGs, such as the latest ROS-based systems, have also been evaluated with random task sets.
Tang et al. \cite{tang2020response} allocate utilization to each chain based on the total utilization of the entire system and the number of chains, and then assign the utilization to the execution units in the chain.
Choi et al. \cite{choi2021picas} similarly assign a utilization to each chain using the UUniFast method from the total system-wide utilization.
Multi-rate DAGs based on such chains can be generated flexibly with the {\it Chain-based} method in MRDAG-Gen.
In addition, since MRDAG-Gen also provides the functionality to automatically set the utilization corresponding to the chain, researchers do not need to implement it on their own.


\section{Conclusion}
\label{sec: conclusion}

In this paper, we proposed a multi-rate DAG generator MRDAG-Gen that covers single-rate DAGs and state-of-the-art multi-rate DAGs.
MRDAG-Gen extended the existing random graph generation method for DAGs and also provided a new chain-based method.
MRDAG-Gen automatically set complex properties such as CCR and utilization.
Moreover, MRDAG-Gen supported researchers by providing functions such as batch generation of all DAG sets for different parameters.
Case studies showed that MRDAG-Gen meets the requirements of DAG studies in a variety of problem settings.
Therefore, MRDAG-Gen provided reliability and easy reproducibility for DAG studies.
In future work, We plan to extend MRDAG-Gen to cover more graph generation methods and complex properties.


%\section*{Acknowledgment}
%\todo{}

\bibliography{./bibliography/master_reference}

\end{document}
