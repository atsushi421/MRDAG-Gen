\section{Evaluation}
\label{sec: evaluation}


\begin{table}[tb]
    \centering
    \caption{Processes that must be implemented by the user \\ and lines of code}
    \label{tab: processes}
    \renewcommand{\arraystretch}{1.2}
    \scalebox{0.8}{
        \begin{tabular}{c|l|c}\hline\hline
            Symbols & \MC{1}{c|}{Processes}                                         & Lines of code \\\hline\hline
            $(a)$   & Loop with different parameter values                          & 3             \\\hline
            $(b)$   & Separate DAGs generated for each parameter into directories   & 5             \\\hline
            $(c)$   & Load .tgff file as DAG                                        & 40            \\\hline
            $(d)$   & Construct a DAG using G(n, p) method                          & 10            \\\hline
            $(e)$   & Construct a DAG using chain-based method                      & 80            \\\hline
            $(f)$   & Merge the exit nodes to a specific number                     & 10            \\\hline
            $(g)$   & Add edges to be weakly connected                              & 15            \\\hline
            $(h)$   & Set the properties randomly within a specific range           & 10            \\\hline
            $(i)$   & \tabml{Set the execution time for each node and                               \\ the communication time for each edge based on the CCR}                             & 25            \\\hline
            $(j)$   & \tabml{Set the utilization for a multi-rate DAG consisting of                 \\ only timer-driven nodes (lines 1-7 in Algorithm~\ref{alg: set_utilization})}                                             & 30            \\\hline
            $(k)$   & \tabml{Set the utilization for a chain-based multi-rate DAG                   \\ (lines 8-19 in Algorithm~\ref{alg: set_utilization})}                                             & 50            \\\hline
        \end{tabular}
    }
\end{table}


This section shows that compared to existing random DAG generation tools, RD-Gen can generate a random DAG set with fewer lines of code and no unique user implementation.
Table~\ref{tab: processes} shows the processes that must be implemented by the user and the number of lines of code.
Here, the process in Table~\ref{tab: processes} is written in Python and shell scripts, and the Python NetworkX library is used for processes related to DAGs.


\begin{table}[tb]
    \centering
    \caption{Effort required to generate random DAG sets \\ for {\it Case Study 1} \cite{subbaraj2020multi, liu2016minimizing, sheikh2016sixteen}}
    \label{tab: eval_case_study_single}
    \renewcommand{\arraystretch}{1.2}
    \scalebox{0.8}{
        \begin{tabular}{|c|c|l|}
            \hline
            \multicolumn{1}{|l|}{}         & \multicolumn{1}{c|}{\begin{tabular}[c]{@{}c@{}}Lines of description\\ for tool options\end{tabular}} & \multicolumn{1}{c|}{Lines of code}                  \\ \hline
            TGFF \cite{tgff}               & 6                                                                                                    & \tabml{$(a) \times 2 + (b) + (c) + (f) + (g) + (i)$ \\ $= 101$} \\ \hline
            GGen \cite{cordeiro2010random} & 3                                                                                                    & \tabml{$(a) \times 2 + (b) + (g) + (i)$             \\ $= 51$}              \\ \hline
            RD-Gen                         & 20                                                                                                   & 0                                                   \\ \hline
        \end{tabular}
    }
\end{table}


The amount of description required to generate a random DAG set for the case studies in Section~\ref{ssec: case_study_1} is shown in Table~\ref{tab: eval_case_study_single}.
In this case, while TGFF and GGen require the user to implement more than 50 lines, RD-Gen can generate all random DAG sets using only the tool's functionality.
Since TGFF and GGen do not have the functionality to generate DAGs with different parameter settings at once, the user must execute the generation command many times while changing the parameters, using shell scripts or other means.
In contrast, RD-Gen generates DAGs with all parameter combinations by specifying {\it Combination} as a parameter and automatically divides them into directories.
Although TGFF uses the {\it Fan-in/Fan-out} method to construct DAGs, the .tgff files output by TGFF are in a proprietary format and require users to implement a reading process.
Furthermore, since TGFF and GGen do not guarantee that the graph is weakly connected, users may unintentionally use a non weakly connected DAG for evaluation.
TGFF and GGen cannot specify parameters such as CCR that are calculated from multiple properties. Therefore, the CCR must be adjusted for the DAG set after it is generated.
In RD-Gen, users can automate all of the above processes by inputting the YAML file as shown on the left side of Fig.~\ref{fig: case_study_single}.


\begin{table}[tb]
    \centering
    \caption{Effort required to generate random DAG sets \\ for {\it Case Study 2} \cite{he2021response, gunzel2021suspension, ueter2021hard}}
    \label{tab: eval_case_study_all_timer}
    \renewcommand{\arraystretch}{1.2}
    \scalebox{0.8}{
        \begin{tabular}{|c|c|l|}
            \hline
            \multicolumn{1}{|l|}{}         & \multicolumn{1}{c|}{\begin{tabular}[c]{@{}c@{}}Lines of description\\ for tool options\end{tabular}} & \multicolumn{1}{c|}{Lines of code}   \\ \hline
            TGFF \cite{tgff}               & -                                                                                                    & \tabml{$(a) + (b) + (d) + (g) + (j)$ \\ $= 63$} \\ \hline
            GGen \cite{cordeiro2010random} & 2                                                                                                    & \tabml{$(a) + (b) + (g) + (j)$       \\ $= 53$}       \\ \hline
            RD-Gen                         & 20                                                                                                   & 0                                    \\ \hline
        \end{tabular}
    }
\end{table}


The effort required to generate a random DAG set for the case studies in Section~\ref{ssec: case_study_2} is shown in Table~\ref{tab: eval_case_study_all_timer}.
TGFF does not support the {\it G(n, p)} method and multi-rate DAG.
GGen allows properties to be randomly set to nodes or edges within a user-specified range after graph construction.
However, GGen does not allow users to specify the total utilization used in the evaluation of most studies that consider multi-rate DAGs.
Therefore, the user has to go through the trouble of implementing such as lines 1-7 in Algorithm~\ref{alg: set_utilization}.
In contrast, RD-Gen uses the UUniFast method to randomly and automatically set the period and execution time for each node to meet the specified total utilization.


\begin{table}[tb]
    \centering
    \caption{Effort required to generate random DAG sets \\ for {\it Case Study 3} \cite{tang2020response, choi2021picas}}
    \label{tab: eval_case_study_chain}
    \renewcommand{\arraystretch}{1.2}
    \scalebox{0.8}{
        \begin{tabular}{|c|c|l|}
            \hline
            \multicolumn{1}{|l|}{}         & \multicolumn{1}{c|}{\begin{tabular}[c]{@{}c@{}}Lines of description\\ for tool options\end{tabular}} & \multicolumn{1}{c|}{Lines of code} \\ \hline
            TGFF \cite{tgff}               & -                                                                                                    & \tabml{$(a) + (b) + (d) + (k)$     \\ $= 138$}      \\ \hline
            GGen \cite{cordeiro2010random} & -                                                                                                    & \tabml{$(a) + (b) + (d) + (k)$     \\ $= 138$}      \\ \hline
            RD-Gen                         & 22                                                                                                   & 0                                  \\ \hline
        \end{tabular}
    }
\end{table}


The amount of description to generate a chain-based random DAG set, as in the case study in Section~\ref{ssec: case_study_3}, is shown in Table~\ref{tab: eval_case_study_chain}.
Chain-based multi-rate DAGs cannot be generated by TGFF and GGen because they are considered in state-of-the-art self-driving systems research.
RD-Gen can generate a batch of random chain-based DAG sets with different total utilization without any user implementation by inputting a YAML file as shown in Fig.~\ref{fig: case_study_chain}.

The evaluation results demonstrated that RD-Gen can generate the random DAG sets used in the evaluation with only an intuitive YAML file description, regardless of the single rate multi rate.
Therefore, RD-Gen is a flexible evaluation platform that can be adapted to various requirements and provides reliability and reproducibility for the latest DAG studies.
