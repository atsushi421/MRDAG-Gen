\section{Related work}
\label{sec: related_work}

This section describes existing random DAG generation tools and existing studies using random DAGs and compares them to MRDAG-Gen.
Table~\ref{tab: comparison} compares MRDAG-Gen and existing methods.

\begin{table}[tb]
    \centering
    {

        \caption{MRDAG-Gen vs existing methods}
        \label{tab: comparison}
        \renewcommand{\arraystretch}{1.2}
        \scalebox{0.8}{
            \begin{tabular}{c|cccccc} \hline\hline
                                                           & RMD & RPU & RDT & RTE & OCG \\\hline
                DAGEN \cite{amalarethinam2011dagen}        &     &     & \ch &     &     \\\hline
                MRTG \cite{ashish2016modular}              &     &     & \ch &     &     \\\hline
                TGFF \cite{tgff}                           &     &     & \ch &     &     \\\hline
                GGen \cite{cordeiro2010random}             & \ch &     & \ch &     &     \\\hline
                Kordon et al. \cite{kordon2020evaluation}  & \ch &     &     &     &     \\\hline
                Yang et al.   \cite{yang2020mixed}         & \ch & \ch &     &     &     \\\hline
                Gunzel et al. \cite{gunzel2021suspension}  & \ch & \ch &     &     &     \\\hline
                Ueter et al. \cite{ueter2021hard}          & \ch & \ch &     &     &     \\\hline
                Dong et al. \cite{dong2019efficient}       & \ch & \ch &     &     &     \\\hline
                He et al. \cite{he2021response}            & \ch & \ch &     &     &     \\\hline
                Voronov et al. \cite{voronov2021ai}        & \ch & \ch &     &     &     \\\hline
                Verucchi et al. \cite{verucchi2020latency} & \ch & \ch &     &     &     \\\hline
                Klaus et al. \cite{klaus2021constrained}   & \ch & \ch &     &     &     \\\hline
                Tang et al. \cite{tang2020response}        & \ch & \ch &     & \ch &     \\\hline
                Choi et al. \cite{choi2021picas}           & \ch & \ch &     & \ch &     \\\hline
                MRDAG-Gen                                  & \ch & \ch & \ch & \ch & \ch \\\hline
            \end{tabular}
        }
        \begin{tablenotes}[normal]{
                \item {RMD}: Random generation of multi-rate DAGs
                \item {RPU}: Random property settings based on total utilization
                \item {RDT}: Random DAG generation tool
                \item {RTE}: Random generation of chain-based multi-rate DAG
                \item {OCG}: One-command batch generation of random DAGs
            }
        \end{tablenotes}
    }
\end{table}


\subsection{Random DAG generation with unique implementation}
\label{sec: random_tool}

The random DAG generation tool provides reliability and reproducibility for the evaluation of scheduling and latency analysis studies.
TGFF \cite{tgff} is the first tool proposed for this purpose and has been used to evaluate the most recent studies \cite{roeder2021energy, fard2021analytical, wu2021evolutionary, costa2021extracting}.
TGFF can determine the shape of the DAG mainly by specifying the maximum and/or minimum input degree and maximum (first) output degree for one node ({\it Fan-in/Fan-out} method).
TGFF can quickly generate many DAGs, and the task set can be easily reproduced by other researchers by inputting the same parameters.
However, TGFF is a tool released in 1998 and has many problems, such as its output format (.tgff), which is difficult to handle and cannot generate the multi-rate DAGs.

GGen \cite{cordeiro2010random} is a unified implementation of the classical task graph generation methods used in the scheduling domain.
GGen allows the user to add properties such as execution period and communication time to nodes and edges after generating DAGs using a user-specified generation method.
However, GGen does not allow the specification of constraints between different properties, such as implicit deadlines (i.e., the execution time of a node must not exceed its period).
Therefore, users with such requirements must adjust the values themselves.

Other random DAG generation tools such as DAGGEN \cite{amalarethinam2011dagen} and MRTG \cite{ashish2016modular} have been proposed.
DAGGEN generates random workflow applications by specifying node load balancing, edge connection probability, and workflow shape.
MRTG is a random DAG generation tool with a module-based implementation for user extensibility.
However, these tools are not capable of generating multi-rate DAGs.
In contrast, MRDAG-Gen can flexibly generate multi-rate DAGs of various types.
In addition, MRDAG-Gen can automatically set properties calculated by multiple values such as CCR and total utilization.


\subsection{Unique implementation of random DAG generation}

This section presents existing studies in which the authors generate random DAG sets for evaluation in their own implemented algorithms and settings.
Since there are no random DAG generation tools capable of generating multi-rate DAGs as described in Section~\ref{sec: random_tool}, most studies that consider multi-rate DAGs have unique implementations.

In real-time systems such as in-vehicle systems and self-driving systems, multi-rate DAGs consisting of only timer-driven nodes are considered.
Many real-time researchers use the {\it G(n, p)} method to construct DAGs and generate random task sets with the different number of nodes, different total utilization, and different periods. \cite{voronov2021ai, he2021response, dong2019efficient}.
While proprietary algorithms are used to construct the graph in some cases, the approach is similar in that total utilization is determined using the UUniFast method and WCET of nodes are assigned based on utilization and periods \cite{yang2020mixed, gunzel2021suspension, ueter2021hard}.

In a multi-rate DAG considering automotive systems, a period is randomly assigned to each node based on the period observed in the automotive application.
Verucchi et al. \cite{verucchi2020latency} randomly extended the automotive benchmark proposed by BOSCH in the 2015 WATERS Challenge \cite{kramer2015real} to analyze the performance of the proposed method.
Verucchi et al. randomly set the task period from $[1, 5, 10, 20, 50, 100, 200, 1000]$ ms, as found in automotive applications, DAG utilization.
Klaus et al. \cite{klaus2021constrained} set the utilization, period, and the number of nodes for each node of the DAG.
Verucchi et al. and Klaus et al. create random task sets based on node chains consisting of only timer-driven nodes.
MRDAG-Gen can also cover such chains consisting of only timer-driven nodes by using {\it Chain-based} methods and specifying the {\it Period type} as {\it "All"}.
Kordon et al. \cite{kordon2020evaluation} randomly set the period, the number of edges per task, release time, and the number of entry nodes for DAGs generated by the Python networkX library.
MRDAG-Gen can automatically set all combinations of total utilization, periods, and the number of nodes as described above.

Studies of chain-based multi-rate DAGs, such as the latest ROS-based systems, have also been evaluated with random task sets.
Tang et al. \cite{tang2020response} allocate utilization to each chain based on the total utilization of the entire system and the number of chains, and then assign the utilization to the execution units in the chain.
Choi et al. \cite{choi2021picas} similarly assign a utilization to each chain using the UUniFast method from the total system-wide utilization.
Multi-rate DAGs based on such chains can be generated flexibly with the {\it Chain-based} method in MRDAG-Gen.
In addition, since MRDAG-Gen also provides the functionality to automatically set the utilization corresponding to the chain, researchers do not need to implement it on their own.
